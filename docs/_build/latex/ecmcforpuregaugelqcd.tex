%% Generated by Sphinx.
\def\sphinxdocclass{report}
\documentclass[letterpaper,10pt,english]{sphinxmanual}
\ifdefined\pdfpxdimen
   \let\sphinxpxdimen\pdfpxdimen\else\newdimen\sphinxpxdimen
\fi \sphinxpxdimen=.75bp\relax
\ifdefined\pdfimageresolution
    \pdfimageresolution= \numexpr \dimexpr1in\relax/\sphinxpxdimen\relax
\fi
%% let collapsible pdf bookmarks panel have high depth per default
\PassOptionsToPackage{bookmarksdepth=5}{hyperref}

\PassOptionsToPackage{booktabs}{sphinx}
\PassOptionsToPackage{colorrows}{sphinx}

\PassOptionsToPackage{warn}{textcomp}
\usepackage[utf8]{inputenc}
\ifdefined\DeclareUnicodeCharacter
% support both utf8 and utf8x syntaxes
  \ifdefined\DeclareUnicodeCharacterAsOptional
    \def\sphinxDUC#1{\DeclareUnicodeCharacter{"#1}}
  \else
    \let\sphinxDUC\DeclareUnicodeCharacter
  \fi
  \sphinxDUC{00A0}{\nobreakspace}
  \sphinxDUC{2500}{\sphinxunichar{2500}}
  \sphinxDUC{2502}{\sphinxunichar{2502}}
  \sphinxDUC{2514}{\sphinxunichar{2514}}
  \sphinxDUC{251C}{\sphinxunichar{251C}}
  \sphinxDUC{2572}{\textbackslash}
\fi
\usepackage{cmap}
\usepackage[T1]{fontenc}
\usepackage{amsmath,amssymb,amstext}
\usepackage{babel}



\usepackage{tgtermes}
\usepackage{tgheros}
\renewcommand{\ttdefault}{txtt}



\usepackage[Bjarne]{fncychap}
\usepackage{sphinx}

\fvset{fontsize=auto}
\usepackage{geometry}


% Include hyperref last.
\usepackage{hyperref}
% Fix anchor placement for figures with captions.
\usepackage{hypcap}% it must be loaded after hyperref.
% Set up styles of URL: it should be placed after hyperref.
\urlstyle{same}

\addto\captionsenglish{\renewcommand{\contentsname}{Contents:}}

\usepackage{sphinxmessages}
\setcounter{tocdepth}{1}



\title{ECMC for pure gauge LQCD}
\date{Mar 04, 2025}
\release{1.0.0}
\author{Yacob Ozdalkiran}
\newcommand{\sphinxlogo}{\vbox{}}
\renewcommand{\releasename}{Release}
\makeindex
\begin{document}

\ifdefined\shorthandoff
  \ifnum\catcode`\=\string=\active\shorthandoff{=}\fi
  \ifnum\catcode`\"=\active\shorthandoff{"}\fi
\fi

\pagestyle{empty}
\sphinxmaketitle
\pagestyle{plain}
\sphinxtableofcontents
\pagestyle{normal}
\phantomsection\label{\detokenize{index::doc}}


\sphinxAtStartPar
Implementation of the Event\sphinxhyphen{}Chain algorithm for SU(3) pure gauge theory on the lattice.
Uses the Wilson action and 1+1 dimensional lattice.



\chapter{src}
\label{\detokenize{modules:src}}\label{\detokenize{modules::doc}}
\sphinxstepscope


\section{gauge\_su3 module}
\label{\detokenize{gauge_su3:module-gauge_su3}}\label{\detokenize{gauge_su3:gauge-su3-module}}\label{\detokenize{gauge_su3::doc}}\index{module@\spxentry{module}!gauge\_su3@\spxentry{gauge\_su3}}\index{gauge\_su3@\spxentry{gauge\_su3}!module@\spxentry{module}}
\sphinxAtStartPar
Useful functions to create/manipulate gauge configurations and SU(3) matrices
\index{calculate\_action() (in module gauge\_su3)@\spxentry{calculate\_action()}\spxextra{in module gauge\_su3}}

\begin{fulllineitems}
\phantomsection\label{\detokenize{gauge_su3:gauge_su3.calculate_action}}
\pysigstartsignatures
\pysiglinewithargsret
{\sphinxcode{\sphinxupquote{gauge\_su3.}}\sphinxbfcode{\sphinxupquote{calculate\_action}}}
{\sphinxparam{\DUrole{n}{conf}}\sphinxparamcomma \sphinxparam{\DUrole{n}{beta}}}
{}
\pysigstopsignatures
\sphinxAtStartPar
Computes the Wilson action of a gauge configuration
\begin{quote}\begin{description}
\sphinxlineitem{Parameters}\begin{itemize}
\item {} 
\sphinxAtStartPar
\sphinxstyleliteralstrong{\sphinxupquote{conf}} (\sphinxstyleliteralemphasis{\sphinxupquote{numpy.array}}) \textendash{} gauge configuration

\item {} 
\sphinxAtStartPar
\sphinxstyleliteralstrong{\sphinxupquote{beta}} (\sphinxstyleliteralemphasis{\sphinxupquote{double}}) \textendash{} inverse coupling constant of the Wilson action

\end{itemize}

\sphinxlineitem{Returns}
\sphinxAtStartPar
the Wilson action of the gauge configuration

\sphinxlineitem{Return type}
\sphinxAtStartPar
double

\end{description}\end{quote}

\end{fulllineitems}

\index{calculate\_diff\_action() (in module gauge\_su3)@\spxentry{calculate\_diff\_action()}\spxextra{in module gauge\_su3}}

\begin{fulllineitems}
\phantomsection\label{\detokenize{gauge_su3:gauge_su3.calculate_diff_action}}
\pysigstartsignatures
\pysiglinewithargsret
{\sphinxcode{\sphinxupquote{gauge\_su3.}}\sphinxbfcode{\sphinxupquote{calculate\_diff\_action}}}
{\sphinxparam{\DUrole{n}{conf}}\sphinxparamcomma \sphinxparam{\DUrole{n}{x}}\sphinxparamcomma \sphinxparam{\DUrole{n}{t}}\sphinxparamcomma \sphinxparam{\DUrole{n}{mu}}\sphinxparamcomma \sphinxparam{\DUrole{n}{m}}\sphinxparamcomma \sphinxparam{\DUrole{n}{beta}}}
{}
\pysigstopsignatures
\sphinxAtStartPar
Computes delta S = new action \sphinxhyphen{} old action the action difference where the x,t,mu link is multiplied by m
\begin{quote}\begin{description}
\sphinxlineitem{Parameters}\begin{itemize}
\item {} 
\sphinxAtStartPar
\sphinxstyleliteralstrong{\sphinxupquote{conf}} (\sphinxstyleliteralemphasis{\sphinxupquote{numpy.array}}) \textendash{} gauge configuration

\item {} 
\sphinxAtStartPar
\sphinxstyleliteralstrong{\sphinxupquote{x}} (\sphinxstyleliteralemphasis{\sphinxupquote{int}}) \textendash{} space coordinate of the link

\item {} 
\sphinxAtStartPar
\sphinxstyleliteralstrong{\sphinxupquote{t}} (\sphinxstyleliteralemphasis{\sphinxupquote{int}}) \textendash{} time coordinate of the link

\item {} 
\sphinxAtStartPar
\sphinxstyleliteralstrong{\sphinxupquote{mu}} (\sphinxstyleliteralemphasis{\sphinxupquote{int}}) \textendash{} lorentz coordinate of the link

\item {} 
\sphinxAtStartPar
\sphinxstyleliteralstrong{\sphinxupquote{m}} (\sphinxstyleliteralemphasis{\sphinxupquote{numpy.array}}) \textendash{} updating SU(3) matrix

\item {} 
\sphinxAtStartPar
\sphinxstyleliteralstrong{\sphinxupquote{beta}} (\sphinxstyleliteralemphasis{\sphinxupquote{double}}) \textendash{} Wilson action inverse coupling constant

\end{itemize}

\sphinxlineitem{Returns}
\sphinxAtStartPar
the difference of action

\sphinxlineitem{Return type}
\sphinxAtStartPar
double

\end{description}\end{quote}

\end{fulllineitems}

\index{calculate\_plaquette() (in module gauge\_su3)@\spxentry{calculate\_plaquette()}\spxextra{in module gauge\_su3}}

\begin{fulllineitems}
\phantomsection\label{\detokenize{gauge_su3:gauge_su3.calculate_plaquette}}
\pysigstartsignatures
\pysiglinewithargsret
{\sphinxcode{\sphinxupquote{gauge\_su3.}}\sphinxbfcode{\sphinxupquote{calculate\_plaquette}}}
{\sphinxparam{\DUrole{n}{gauge\_config}}\sphinxparamcomma \sphinxparam{\DUrole{n}{x}}\sphinxparamcomma \sphinxparam{\DUrole{n}{t}}}
{}
\pysigstopsignatures
\sphinxAtStartPar
Computes the plaquette at a given lattice site (x,t,mu). The operation is : U(x,t,0) U(x+1,t,1) U*(x,t+1,0)\textasciicircum{}T U*(x,t,1)\textasciicircum{}T
\begin{quote}\begin{description}
\sphinxlineitem{Parameters}\begin{itemize}
\item {} 
\sphinxAtStartPar
\sphinxstyleliteralstrong{\sphinxupquote{gauge\_config}} (\sphinxstyleliteralemphasis{\sphinxupquote{numpy.array}}) \textendash{} gauge configuration

\item {} 
\sphinxAtStartPar
\sphinxstyleliteralstrong{\sphinxupquote{x}} (\sphinxstyleliteralemphasis{\sphinxupquote{int}}) \textendash{} space coordinate of the link

\item {} 
\sphinxAtStartPar
\sphinxstyleliteralstrong{\sphinxupquote{t}} (\sphinxstyleliteralemphasis{\sphinxupquote{int}}) \textendash{} time coordinate of the link

\end{itemize}

\sphinxlineitem{Returns}
\sphinxAtStartPar
the plaquette (3x3 matrix)

\sphinxlineitem{Return type}
\sphinxAtStartPar
numpy.array

\end{description}\end{quote}

\end{fulllineitems}

\index{conf\_est\_SU3() (in module gauge\_su3)@\spxentry{conf\_est\_SU3()}\spxextra{in module gauge\_su3}}

\begin{fulllineitems}
\phantomsection\label{\detokenize{gauge_su3:gauge_su3.conf_est_SU3}}
\pysigstartsignatures
\pysiglinewithargsret
{\sphinxcode{\sphinxupquote{gauge\_su3.}}\sphinxbfcode{\sphinxupquote{conf\_est\_SU3}}}
{\sphinxparam{\DUrole{n}{conf}}\sphinxparamcomma \sphinxparam{\DUrole{n}{tol}\DUrole{o}{=}\DUrole{default_value}{1e\sphinxhyphen{}10}}}
{}
\pysigstopsignatures
\sphinxAtStartPar
Checks wheter all the matrices in a lattice gauge configuration are SU(3) matrices
\begin{quote}\begin{description}
\sphinxlineitem{Parameters}\begin{itemize}
\item {} 
\sphinxAtStartPar
\sphinxstyleliteralstrong{\sphinxupquote{conf}} (\sphinxstyleliteralemphasis{\sphinxupquote{numpy.array}}) \textendash{} The gauge configuration to check

\item {} 
\sphinxAtStartPar
\sphinxstyleliteralstrong{\sphinxupquote{tol}} (\sphinxstyleliteralemphasis{\sphinxupquote{double}}\sphinxstyleliteralemphasis{\sphinxupquote{, }}\sphinxstyleliteralemphasis{\sphinxupquote{optional}}) \textendash{} Tolerance of numerical checks. Defaults to 1e\sphinxhyphen{}10.

\end{itemize}

\sphinxlineitem{Returns}
\sphinxAtStartPar
True if all the matrices in a lattice gauge configuration are SU(3) matrices, if not False

\sphinxlineitem{Return type}
\sphinxAtStartPar
bool

\end{description}\end{quote}

\end{fulllineitems}

\index{el() (in module gauge\_su3)@\spxentry{el()}\spxextra{in module gauge\_su3}}

\begin{fulllineitems}
\phantomsection\label{\detokenize{gauge_su3:gauge_su3.el}}
\pysigstartsignatures
\pysiglinewithargsret
{\sphinxcode{\sphinxupquote{gauge\_su3.}}\sphinxbfcode{\sphinxupquote{el}}}
{\sphinxparam{\DUrole{n}{xi}}\sphinxparamcomma \sphinxparam{\DUrole{n}{index}}}
{}
\pysigstopsignatures
\sphinxAtStartPar
“”Generates the exp(xi*lambda\_i) where lambda\_i is the i\_th Gell\sphinxhyphen{}Mann matrice
\begin{quote}\begin{description}
\sphinxlineitem{Parameters}\begin{itemize}
\item {} 
\sphinxAtStartPar
\sphinxstyleliteralstrong{\sphinxupquote{xi}} (\sphinxstyleliteralemphasis{\sphinxupquote{double}}) \textendash{} angle

\item {} 
\sphinxAtStartPar
\sphinxstyleliteralstrong{\sphinxupquote{index}} (\sphinxstyleliteralemphasis{\sphinxupquote{int}}) \textendash{} index of the Gell\sphinxhyphen{}Mann matrice, can only be 2,3,5 or 8

\end{itemize}

\sphinxlineitem{Raises}
\sphinxAtStartPar
\sphinxstyleliteralstrong{\sphinxupquote{ValueError}} \textendash{} if index is not 2,3,5 or 8

\sphinxlineitem{Returns}
\sphinxAtStartPar
exp(xi*lambda\_i)

\sphinxlineitem{Return type}
\sphinxAtStartPar
numpy.array

\end{description}\end{quote}

\end{fulllineitems}

\index{el\_2() (in module gauge\_su3)@\spxentry{el\_2()}\spxextra{in module gauge\_su3}}

\begin{fulllineitems}
\phantomsection\label{\detokenize{gauge_su3:gauge_su3.el_2}}
\pysigstartsignatures
\pysiglinewithargsret
{\sphinxcode{\sphinxupquote{gauge\_su3.}}\sphinxbfcode{\sphinxupquote{el\_2}}}
{\sphinxparam{\DUrole{n}{xi}}}
{}
\pysigstopsignatures
\sphinxAtStartPar
Generates exp(xi*lambda\_2) where lambda\_2 is the 2nd Gell\sphinxhyphen{}Man matrice
\begin{quote}\begin{description}
\sphinxlineitem{Parameters}
\sphinxAtStartPar
\sphinxstyleliteralstrong{\sphinxupquote{xi}} (\sphinxstyleliteralemphasis{\sphinxupquote{double}}) \textendash{} angle

\sphinxlineitem{Returns}
\sphinxAtStartPar
exp(xi*lambda\_2)

\sphinxlineitem{Return type}
\sphinxAtStartPar
numpy.array

\end{description}\end{quote}

\end{fulllineitems}

\index{el\_3() (in module gauge\_su3)@\spxentry{el\_3()}\spxextra{in module gauge\_su3}}

\begin{fulllineitems}
\phantomsection\label{\detokenize{gauge_su3:gauge_su3.el_3}}
\pysigstartsignatures
\pysiglinewithargsret
{\sphinxcode{\sphinxupquote{gauge\_su3.}}\sphinxbfcode{\sphinxupquote{el\_3}}}
{\sphinxparam{\DUrole{n}{xi}}}
{}
\pysigstopsignatures
\sphinxAtStartPar
Generates exp(xi*lambda\_3) where lambda\_3 is the 3rd Gell\sphinxhyphen{}Man matrice
\begin{quote}\begin{description}
\sphinxlineitem{Parameters}
\sphinxAtStartPar
\sphinxstyleliteralstrong{\sphinxupquote{xi}} (\sphinxstyleliteralemphasis{\sphinxupquote{double}}) \textendash{} angle

\sphinxlineitem{Returns}
\sphinxAtStartPar
exp(xi*lambda\_3)

\sphinxlineitem{Return type}
\sphinxAtStartPar
numpy.array

\end{description}\end{quote}

\end{fulllineitems}

\index{el\_5() (in module gauge\_su3)@\spxentry{el\_5()}\spxextra{in module gauge\_su3}}

\begin{fulllineitems}
\phantomsection\label{\detokenize{gauge_su3:gauge_su3.el_5}}
\pysigstartsignatures
\pysiglinewithargsret
{\sphinxcode{\sphinxupquote{gauge\_su3.}}\sphinxbfcode{\sphinxupquote{el\_5}}}
{\sphinxparam{\DUrole{n}{xi}}}
{}
\pysigstopsignatures
\sphinxAtStartPar
Generates exp(xi*lambda\_5) where lambda\_5 is the 5th Gell\sphinxhyphen{}Man matrice
\begin{quote}\begin{description}
\sphinxlineitem{Parameters}
\sphinxAtStartPar
\sphinxstyleliteralstrong{\sphinxupquote{xi}} (\sphinxstyleliteralemphasis{\sphinxupquote{double}}) \textendash{} angle

\sphinxlineitem{Returns}
\sphinxAtStartPar
exp(xi*lambda\_5)

\sphinxlineitem{Return type}
\sphinxAtStartPar
numpy.array

\end{description}\end{quote}

\end{fulllineitems}

\index{el\_8() (in module gauge\_su3)@\spxentry{el\_8()}\spxextra{in module gauge\_su3}}

\begin{fulllineitems}
\phantomsection\label{\detokenize{gauge_su3:gauge_su3.el_8}}
\pysigstartsignatures
\pysiglinewithargsret
{\sphinxcode{\sphinxupquote{gauge\_su3.}}\sphinxbfcode{\sphinxupquote{el\_8}}}
{\sphinxparam{\DUrole{n}{xi}}}
{}
\pysigstopsignatures
\sphinxAtStartPar
Generates exp(xi*lambda\_8) where lambda\_8 is the 8th Gell\sphinxhyphen{}Man matrice
\begin{quote}\begin{description}
\sphinxlineitem{Parameters}
\sphinxAtStartPar
\sphinxstyleliteralstrong{\sphinxupquote{xi}} (\sphinxstyleliteralemphasis{\sphinxupquote{double}}) \textendash{} angle

\sphinxlineitem{Returns}
\sphinxAtStartPar
exp(xi*lambda\_8)

\sphinxlineitem{Return type}
\sphinxAtStartPar
numpy.array

\end{description}\end{quote}

\end{fulllineitems}

\index{est\_SU3() (in module gauge\_su3)@\spxentry{est\_SU3()}\spxextra{in module gauge\_su3}}

\begin{fulllineitems}
\phantomsection\label{\detokenize{gauge_su3:gauge_su3.est_SU3}}
\pysigstartsignatures
\pysiglinewithargsret
{\sphinxcode{\sphinxupquote{gauge\_su3.}}\sphinxbfcode{\sphinxupquote{est\_SU3}}}
{\sphinxparam{\DUrole{n}{U}}\sphinxparamcomma \sphinxparam{\DUrole{n}{tol}\DUrole{o}{=}\DUrole{default_value}{1e\sphinxhyphen{}10}}}
{}
\pysigstopsignatures
\sphinxAtStartPar
Checks wether a matrix is in SU(3)
\begin{quote}\begin{description}
\sphinxlineitem{Parameters}\begin{itemize}
\item {} 
\sphinxAtStartPar
\sphinxstyleliteralstrong{\sphinxupquote{U}} (\sphinxstyleliteralemphasis{\sphinxupquote{numpy.array}}) \textendash{} 3x3 matrix to check

\item {} 
\sphinxAtStartPar
\sphinxstyleliteralstrong{\sphinxupquote{tol}} (\sphinxstyleliteralemphasis{\sphinxupquote{double}}\sphinxstyleliteralemphasis{\sphinxupquote{, }}\sphinxstyleliteralemphasis{\sphinxupquote{optional}}) \textendash{} Tolerance of numerical checks. Defaults to 1e\sphinxhyphen{}10.

\end{itemize}

\sphinxlineitem{Returns}
\sphinxAtStartPar
True is the matrix is in SU(3), False if not

\sphinxlineitem{Return type}
\sphinxAtStartPar
bool

\end{description}\end{quote}

\end{fulllineitems}

\index{get\_link() (in module gauge\_su3)@\spxentry{get\_link()}\spxextra{in module gauge\_su3}}

\begin{fulllineitems}
\phantomsection\label{\detokenize{gauge_su3:gauge_su3.get_link}}
\pysigstartsignatures
\pysiglinewithargsret
{\sphinxcode{\sphinxupquote{gauge\_su3.}}\sphinxbfcode{\sphinxupquote{get\_link}}}
{\sphinxparam{\DUrole{n}{conf}}\sphinxparamcomma \sphinxparam{\DUrole{n}{x}}\sphinxparamcomma \sphinxparam{\DUrole{n}{t}}\sphinxparamcomma \sphinxparam{\DUrole{n}{mu}}}
{}
\pysigstopsignatures
\sphinxAtStartPar
Gets a link in a gauge configuration with periodic boundary conditions
\begin{quote}\begin{description}
\sphinxlineitem{Parameters}\begin{itemize}
\item {} 
\sphinxAtStartPar
\sphinxstyleliteralstrong{\sphinxupquote{conf}} (\sphinxstyleliteralemphasis{\sphinxupquote{numpy.array}}) \textendash{} gauge configuration

\item {} 
\sphinxAtStartPar
\sphinxstyleliteralstrong{\sphinxupquote{x}} (\sphinxstyleliteralemphasis{\sphinxupquote{int}}) \textendash{} space coordinate of the link

\item {} 
\sphinxAtStartPar
\sphinxstyleliteralstrong{\sphinxupquote{t}} (\sphinxstyleliteralemphasis{\sphinxupquote{int}}) \textendash{} time coordinate of the link

\item {} 
\sphinxAtStartPar
\sphinxstyleliteralstrong{\sphinxupquote{mu}} (\sphinxstyleliteralemphasis{\sphinxupquote{int}}) \textendash{} lorentz coordinate of the link

\end{itemize}

\sphinxlineitem{Returns}
\sphinxAtStartPar
the matrix of the link

\sphinxlineitem{Return type}
\sphinxAtStartPar
numpy.array

\end{description}\end{quote}

\end{fulllineitems}

\index{init\_conf() (in module gauge\_su3)@\spxentry{init\_conf()}\spxextra{in module gauge\_su3}}

\begin{fulllineitems}
\phantomsection\label{\detokenize{gauge_su3:gauge_su3.init_conf}}
\pysigstartsignatures
\pysiglinewithargsret
{\sphinxcode{\sphinxupquote{gauge\_su3.}}\sphinxbfcode{\sphinxupquote{init\_conf}}}
{\sphinxparam{\DUrole{n}{L}\DUrole{o}{=}\DUrole{default_value}{3}}\sphinxparamcomma \sphinxparam{\DUrole{n}{T}\DUrole{o}{=}\DUrole{default_value}{3}}\sphinxparamcomma \sphinxparam{\DUrole{n}{cold}\DUrole{o}{=}\DUrole{default_value}{True}}}
{}
\pysigstopsignatures
\sphinxAtStartPar
Returns a 1+1d gauge configuration of size L*T
\begin{quote}\begin{description}
\sphinxlineitem{Parameters}\begin{itemize}
\item {} 
\sphinxAtStartPar
\sphinxstyleliteralstrong{\sphinxupquote{L}} (\sphinxstyleliteralemphasis{\sphinxupquote{int}}\sphinxstyleliteralemphasis{\sphinxupquote{, }}\sphinxstyleliteralemphasis{\sphinxupquote{optional}}) \textendash{} Spatial dimension of the lattice. Defaults to 3.

\item {} 
\sphinxAtStartPar
\sphinxstyleliteralstrong{\sphinxupquote{T}} (\sphinxstyleliteralemphasis{\sphinxupquote{int}}\sphinxstyleliteralemphasis{\sphinxupquote{, }}\sphinxstyleliteralemphasis{\sphinxupquote{optional}}) \textendash{} Temporal dimension of the lattice. Defaults to 3.

\item {} 
\sphinxAtStartPar
\sphinxstyleliteralstrong{\sphinxupquote{cold}} (\sphinxstyleliteralemphasis{\sphinxupquote{bool}}\sphinxstyleliteralemphasis{\sphinxupquote{, }}\sphinxstyleliteralemphasis{\sphinxupquote{optional}}) \textendash{} If True, all matrix initialized to identity, else all matrices are random. Defaults to True.

\end{itemize}

\sphinxlineitem{Returns}
\sphinxAtStartPar
L*T 1+1d lattice gauge configuration

\sphinxlineitem{Return type}
\sphinxAtStartPar
numpy.array

\end{description}\end{quote}

\end{fulllineitems}

\index{mat\_rand\_su3() (in module gauge\_su3)@\spxentry{mat\_rand\_su3()}\spxextra{in module gauge\_su3}}

\begin{fulllineitems}
\phantomsection\label{\detokenize{gauge_su3:gauge_su3.mat_rand_su3}}
\pysigstartsignatures
\pysiglinewithargsret
{\sphinxcode{\sphinxupquote{gauge\_su3.}}\sphinxbfcode{\sphinxupquote{mat\_rand\_su3}}}
{}
{}
\pysigstopsignatures
\sphinxAtStartPar
Generates a random SU(3) matrice using the parametrization of \sphinxurl{https://arxiv.org/abs/physics/9708015}
\begin{quote}\begin{description}
\sphinxlineitem{Returns}
\sphinxAtStartPar
random SU(3) matrix

\sphinxlineitem{Return type}
\sphinxAtStartPar
numpy.array

\end{description}\end{quote}

\end{fulllineitems}

\index{random() (in module gauge\_su3)@\spxentry{random()}\spxextra{in module gauge\_su3}}

\begin{fulllineitems}
\phantomsection\label{\detokenize{gauge_su3:gauge_su3.random}}
\pysigstartsignatures
\pysiglinewithargsret
{\sphinxcode{\sphinxupquote{gauge\_su3.}}\sphinxbfcode{\sphinxupquote{random}}}
{}
{}
\pysigstopsignatures
\end{fulllineitems}

\index{transf\_gauge() (in module gauge\_su3)@\spxentry{transf\_gauge()}\spxextra{in module gauge\_su3}}

\begin{fulllineitems}
\phantomsection\label{\detokenize{gauge_su3:gauge_su3.transf_gauge}}
\pysigstartsignatures
\pysiglinewithargsret
{\sphinxcode{\sphinxupquote{gauge\_su3.}}\sphinxbfcode{\sphinxupquote{transf\_gauge}}}
{\sphinxparam{\DUrole{n}{conf}}}
{}
\pysigstopsignatures
\sphinxAtStartPar
Returns a copy of the gauge configuration where all the links were multiplied by a random SU(3) matrix to simulate a random local gauge transform.
\begin{quote}\begin{description}
\sphinxlineitem{Parameters}
\sphinxAtStartPar
\sphinxstyleliteralstrong{\sphinxupquote{conf}} (\sphinxstyleliteralemphasis{\sphinxupquote{numpy.array}}) \textendash{} gauge configuration

\sphinxlineitem{Returns}
\sphinxAtStartPar
the transformed configuration

\sphinxlineitem{Return type}
\sphinxAtStartPar
numpy.array

\end{description}\end{quote}

\end{fulllineitems}

\index{update\_link() (in module gauge\_su3)@\spxentry{update\_link()}\spxextra{in module gauge\_su3}}

\begin{fulllineitems}
\phantomsection\label{\detokenize{gauge_su3:gauge_su3.update_link}}
\pysigstartsignatures
\pysiglinewithargsret
{\sphinxcode{\sphinxupquote{gauge\_su3.}}\sphinxbfcode{\sphinxupquote{update\_link}}}
{\sphinxparam{\DUrole{n}{conf}}\sphinxparamcomma \sphinxparam{\DUrole{n}{x}}\sphinxparamcomma \sphinxparam{\DUrole{n}{t}}\sphinxparamcomma \sphinxparam{\DUrole{n}{mu}}\sphinxparamcomma \sphinxparam{\DUrole{n}{new\_matrix}}}
{}
\pysigstopsignatures
\sphinxAtStartPar
Updates a gauge configuration by multiplying the selected link with a matrix
\begin{quote}\begin{description}
\sphinxlineitem{Parameters}\begin{itemize}
\item {} 
\sphinxAtStartPar
\sphinxstyleliteralstrong{\sphinxupquote{conf}} (\sphinxstyleliteralemphasis{\sphinxupquote{numpy.array}}) \textendash{} gauge configuration

\item {} 
\sphinxAtStartPar
\sphinxstyleliteralstrong{\sphinxupquote{x}} (\sphinxstyleliteralemphasis{\sphinxupquote{int}}) \textendash{} space coordinate of the link

\item {} 
\sphinxAtStartPar
\sphinxstyleliteralstrong{\sphinxupquote{t}} (\sphinxstyleliteralemphasis{\sphinxupquote{int}}) \textendash{} time coordinate of the link

\item {} 
\sphinxAtStartPar
\sphinxstyleliteralstrong{\sphinxupquote{mu}} (\sphinxstyleliteralemphasis{\sphinxupquote{int}}) \textendash{} lorentz coordinate of the link

\item {} 
\sphinxAtStartPar
\sphinxstyleliteralstrong{\sphinxupquote{new\_matrix}} (\sphinxstyleliteralemphasis{\sphinxupquote{numpy.array}}) \textendash{} the updating matrix

\end{itemize}

\end{description}\end{quote}

\end{fulllineitems}


\sphinxstepscope


\section{analytical\_reject module}
\label{\detokenize{analytical_reject:module-analytical_reject}}\label{\detokenize{analytical_reject:analytical-reject-module}}\label{\detokenize{analytical_reject::doc}}\index{module@\spxentry{module}!analytical\_reject@\spxentry{analytical\_reject}}\index{analytical\_reject@\spxentry{analytical\_reject}!module@\spxentry{module}}
\sphinxAtStartPar
Generation of rejects by analytical resolution of the inversion equation. Available only for lambda\_2, lambda\_3 and lambda\_5.
\index{coeff\_c() (in module analytical\_reject)@\spxentry{coeff\_c()}\spxextra{in module analytical\_reject}}

\begin{fulllineitems}
\phantomsection\label{\detokenize{analytical_reject:analytical_reject.coeff_c}}
\pysigstartsignatures
\pysiglinewithargsret
{\sphinxcode{\sphinxupquote{analytical\_reject.}}\sphinxbfcode{\sphinxupquote{coeff\_c}}}
{\sphinxparam{\DUrole{n}{gamma}}\sphinxparamcomma \sphinxparam{\DUrole{n}{plaquette}}\sphinxparamcomma \sphinxparam{\DUrole{n}{index\_lambda}}}
{}
\pysigstopsignatures
\sphinxAtStartPar
Returns the value of c’ (cf pdf) needed to solve the reject equation
\begin{quote}\begin{description}
\sphinxlineitem{Parameters}\begin{itemize}
\item {} 
\sphinxAtStartPar
\sphinxstyleliteralstrong{\sphinxupquote{gamma}} (\sphinxstyleliteralemphasis{\sphinxupquote{double}}) \textendash{} the real number to reduce

\item {} 
\sphinxAtStartPar
\sphinxstyleliteralstrong{\sphinxupquote{plaquette}} (\sphinxstyleliteralemphasis{\sphinxupquote{numpy.array}}) \textendash{} the plaquette used to compute a,b coefficients

\item {} 
\sphinxAtStartPar
\sphinxstyleliteralstrong{\sphinxupquote{index\_lambda}} (\sphinxstyleliteralemphasis{\sphinxupquote{int}}) \textendash{} the index of the gell\sphinxhyphen{}mann matrix

\end{itemize}

\sphinxlineitem{Raises}\begin{itemize}
\item {} 
\sphinxAtStartPar
\sphinxstyleliteralstrong{\sphinxupquote{ValueError}} \textendash{} if the corresponding (a,b) sign does not match the assessement of foward/backward path (cf tables)

\item {} 
\sphinxAtStartPar
\sphinxstyleliteralstrong{\sphinxupquote{ValueError}} \textendash{} if the corresponding (a,b) sign does not match the assessement of foward/backward path (cf tables)

\end{itemize}

\sphinxlineitem{Returns}
\sphinxAtStartPar
the c’ coefficient

\sphinxlineitem{Return type}
\sphinxAtStartPar
double

\end{description}\end{quote}

\end{fulllineitems}

\index{coeffs() (in module analytical\_reject)@\spxentry{coeffs()}\spxextra{in module analytical\_reject}}

\begin{fulllineitems}
\phantomsection\label{\detokenize{analytical_reject:analytical_reject.coeffs}}
\pysigstartsignatures
\pysiglinewithargsret
{\sphinxcode{\sphinxupquote{analytical\_reject.}}\sphinxbfcode{\sphinxupquote{coeffs}}}
{\sphinxparam{\DUrole{n}{plaquette}}\sphinxparamcomma \sphinxparam{\DUrole{n}{index\_lambda}}}
{}
\pysigstopsignatures
\sphinxAtStartPar
Computes the a,b,c coefficients for a given plaquette and gell\sphinxhyphen{}mann index
\begin{quote}\begin{description}
\sphinxlineitem{Parameters}\begin{itemize}
\item {} 
\sphinxAtStartPar
\sphinxstyleliteralstrong{\sphinxupquote{plaquette}} (\sphinxstyleliteralemphasis{\sphinxupquote{numpy.array}}) \textendash{} plaquette

\item {} 
\sphinxAtStartPar
\sphinxstyleliteralstrong{\sphinxupquote{index\_lambda}} (\sphinxstyleliteralemphasis{\sphinxupquote{int}}) \textendash{} gell\sphinxhyphen{}man index

\end{itemize}

\sphinxlineitem{Raises}
\sphinxAtStartPar
\sphinxstyleliteralstrong{\sphinxupquote{ValueError}} \textendash{} the gell mann index must be 2,3 or 5

\sphinxlineitem{Returns}
\sphinxAtStartPar
the a,b,c coefficients

\sphinxlineitem{Return type}
\sphinxAtStartPar
double,double,double

\end{description}\end{quote}

\end{fulllineitems}

\index{coeffs\_ana() (in module analytical\_reject)@\spxentry{coeffs\_ana()}\spxextra{in module analytical\_reject}}

\begin{fulllineitems}
\phantomsection\label{\detokenize{analytical_reject:analytical_reject.coeffs_ana}}
\pysigstartsignatures
\pysiglinewithargsret
{\sphinxcode{\sphinxupquote{analytical\_reject.}}\sphinxbfcode{\sphinxupquote{coeffs\_ana}}}
{\sphinxparam{\DUrole{n}{gamma}}\sphinxparamcomma \sphinxparam{\DUrole{n}{plaquette}}\sphinxparamcomma \sphinxparam{\DUrole{n}{index\_lambda}}}
{}
\pysigstopsignatures
\sphinxAtStartPar
Takes a gamma, reduces it, checks if it is forward or backwards; if backwards, returns a and b correctly modified according to index\_lambda.
\begin{quote}\begin{description}
\sphinxlineitem{Parameters}\begin{itemize}
\item {} 
\sphinxAtStartPar
\sphinxstyleliteralstrong{\sphinxupquote{gamma}} (\sphinxstyleliteralemphasis{\sphinxupquote{double}}) \textendash{} the real number to reduce

\item {} 
\sphinxAtStartPar
\sphinxstyleliteralstrong{\sphinxupquote{plaquette}} (\sphinxstyleliteralemphasis{\sphinxupquote{numpy.array}}) \textendash{} the plaquette used to compute a,b coefficients

\item {} 
\sphinxAtStartPar
\sphinxstyleliteralstrong{\sphinxupquote{index\_lambda}} (\sphinxstyleliteralemphasis{\sphinxupquote{int}}) \textendash{} the index of the gell\sphinxhyphen{}mann matrix

\end{itemize}

\sphinxlineitem{Returns}
\sphinxAtStartPar
+/\sphinxhyphen{}a,+/\sphinxhyphen{}b according to the case

\sphinxlineitem{Return type}
\sphinxAtStartPar
double, double

\end{description}\end{quote}

\end{fulllineitems}

\index{gamma\_aller\_retour() (in module analytical\_reject)@\spxentry{gamma\_aller\_retour()}\spxextra{in module analytical\_reject}}

\begin{fulllineitems}
\phantomsection\label{\detokenize{analytical_reject:analytical_reject.gamma_aller_retour}}
\pysigstartsignatures
\pysiglinewithargsret
{\sphinxcode{\sphinxupquote{analytical\_reject.}}\sphinxbfcode{\sphinxupquote{gamma\_aller\_retour}}}
{\sphinxparam{\DUrole{n}{gamma}}\sphinxparamcomma \sphinxparam{\DUrole{n}{plaquette}}\sphinxparamcomma \sphinxparam{\DUrole{n}{index\_lambda}}}
{}
\pysigstopsignatures
\sphinxAtStartPar
Takes a gamma, reduces it, and returns True if it is reached during the forward path, False otherwise.
\begin{quote}\begin{description}
\sphinxlineitem{Parameters}\begin{itemize}
\item {} 
\sphinxAtStartPar
\sphinxstyleliteralstrong{\sphinxupquote{gamma}} (\sphinxstyleliteralemphasis{\sphinxupquote{double}}) \textendash{} the real number to reduce

\item {} 
\sphinxAtStartPar
\sphinxstyleliteralstrong{\sphinxupquote{plaquette}} (\sphinxstyleliteralemphasis{\sphinxupquote{numpy.array}}) \textendash{} the plaquette used to compute a,b coefficients

\item {} 
\sphinxAtStartPar
\sphinxstyleliteralstrong{\sphinxupquote{index\_lambda}} (\sphinxstyleliteralemphasis{\sphinxupquote{int}}) \textendash{} the index of the gell\sphinxhyphen{}mann matrix

\end{itemize}

\sphinxlineitem{Returns}
\sphinxAtStartPar
True if gamma is reached during the forward path, False otherwise

\sphinxlineitem{Return type}
\sphinxAtStartPar
bool

\end{description}\end{quote}

\end{fulllineitems}

\index{intervalle() (in module analytical\_reject)@\spxentry{intervalle()}\spxextra{in module analytical\_reject}}

\begin{fulllineitems}
\phantomsection\label{\detokenize{analytical_reject:analytical_reject.intervalle}}
\pysigstartsignatures
\pysiglinewithargsret
{\sphinxcode{\sphinxupquote{analytical\_reject.}}\sphinxbfcode{\sphinxupquote{intervalle}}}
{\sphinxparam{\DUrole{n}{gamma}}\sphinxparamcomma \sphinxparam{\DUrole{n}{plaquette}}\sphinxparamcomma \sphinxparam{\DUrole{n}{index\_lambda}}}
{}
\pysigstopsignatures
\sphinxAtStartPar
Takes a gamma, reduces it, checks if it is reached during a forward or backward path, and returns the interval in which to search for the rejection angle, given that the variable t represents the total displacement (cf pdf for xi parametrization)
\begin{quote}\begin{description}
\sphinxlineitem{Parameters}\begin{itemize}
\item {} 
\sphinxAtStartPar
\sphinxstyleliteralstrong{\sphinxupquote{gamma}} (\sphinxstyleliteralemphasis{\sphinxupquote{double}}) \textendash{} the real number to reduce

\item {} 
\sphinxAtStartPar
\sphinxstyleliteralstrong{\sphinxupquote{plaquette}} (\sphinxstyleliteralemphasis{\sphinxupquote{numpy.array}}) \textendash{} the plaquette used to compute a,b coefficients

\item {} 
\sphinxAtStartPar
\sphinxstyleliteralstrong{\sphinxupquote{index\_lambda}} (\sphinxstyleliteralemphasis{\sphinxupquote{int}}) \textendash{} the index of the gell\sphinxhyphen{}mann matrix

\end{itemize}

\sphinxlineitem{Returns}
\sphinxAtStartPar
the bound of the intervall on which gamma is attained for the t variable (cf pdf for xi(t) parametrization)

\sphinxlineitem{Return type}
\sphinxAtStartPar
(double,double)

\end{description}\end{quote}

\end{fulllineitems}

\index{reduce() (in module analytical\_reject)@\spxentry{reduce()}\spxextra{in module analytical\_reject}}

\begin{fulllineitems}
\phantomsection\label{\detokenize{analytical_reject:analytical_reject.reduce}}
\pysigstartsignatures
\pysiglinewithargsret
{\sphinxcode{\sphinxupquote{analytical\_reject.}}\sphinxbfcode{\sphinxupquote{reduce}}}
{\sphinxparam{\DUrole{n}{gamma}}\sphinxparamcomma \sphinxparam{\DUrole{n}{plaquette}}\sphinxparamcomma \sphinxparam{\DUrole{n}{index\_lambda}}}
{}
\pysigstopsignatures
\sphinxAtStartPar
Returns gamma with the periodic contributions corresponding to index\_lambda removed, see Tables 3 and 4 of the PDF document.
\begin{quote}\begin{description}
\sphinxlineitem{Parameters}\begin{itemize}
\item {} 
\sphinxAtStartPar
\sphinxstyleliteralstrong{\sphinxupquote{gamma}} (\sphinxstyleliteralemphasis{\sphinxupquote{double}}) \textendash{} the real number to reduce

\item {} 
\sphinxAtStartPar
\sphinxstyleliteralstrong{\sphinxupquote{plaquette}} (\sphinxstyleliteralemphasis{\sphinxupquote{numpy.array}}) \textendash{} the plaquette used to compute a,b coefficients

\item {} 
\sphinxAtStartPar
\sphinxstyleliteralstrong{\sphinxupquote{index\_lambda}} (\sphinxstyleliteralemphasis{\sphinxupquote{int}}) \textendash{} the index of the gell\sphinxhyphen{}mann matrix, defines the contribution to remove

\end{itemize}

\sphinxlineitem{Raises}
\sphinxAtStartPar
\sphinxstyleliteralstrong{\sphinxupquote{ValueError}} \textendash{} index\_lambda must be in 2,3,5

\sphinxlineitem{Returns}
\sphinxAtStartPar
reduced gamma

\sphinxlineitem{Return type}
\sphinxAtStartPar
double

\end{description}\end{quote}

\end{fulllineitems}

\index{rejet\_ana() (in module analytical\_reject)@\spxentry{rejet\_ana()}\spxextra{in module analytical\_reject}}

\begin{fulllineitems}
\phantomsection\label{\detokenize{analytical_reject:analytical_reject.rejet_ana}}
\pysigstartsignatures
\pysiglinewithargsret
{\sphinxcode{\sphinxupquote{analytical\_reject.}}\sphinxbfcode{\sphinxupquote{rejet\_ana}}}
{\sphinxparam{\DUrole{n}{gamma}}\sphinxparamcomma \sphinxparam{\DUrole{n}{plaquette}}\sphinxparamcomma \sphinxparam{\DUrole{n}{index\_lambda}}}
{}
\pysigstopsignatures
\sphinxAtStartPar
Takes a gamma, reduces it, checks if it is forward or backward; if backward, removes the forward contribution, then solves the inversion equation in the correct interval and returns the corresponding angle xi(t).
\begin{quote}\begin{description}
\sphinxlineitem{Parameters}\begin{itemize}
\item {} 
\sphinxAtStartPar
\sphinxstyleliteralstrong{\sphinxupquote{gamma}} (\sphinxstyleliteralemphasis{\sphinxupquote{double}}) \textendash{} the real number to reduce

\item {} 
\sphinxAtStartPar
\sphinxstyleliteralstrong{\sphinxupquote{plaquette}} (\sphinxstyleliteralemphasis{\sphinxupquote{numpy.array}}) \textendash{} the plaquette used to compute a,b coefficients

\item {} 
\sphinxAtStartPar
\sphinxstyleliteralstrong{\sphinxupquote{index\_lambda}} (\sphinxstyleliteralemphasis{\sphinxupquote{int}}) \textendash{} the index of the gell\sphinxhyphen{}mann matrix

\end{itemize}

\sphinxlineitem{Raises}\begin{itemize}
\item {} 
\sphinxAtStartPar
\sphinxstyleliteralstrong{\sphinxupquote{ValueError}} \textendash{} if we have 2 solutions in the determined interval

\item {} 
\sphinxAtStartPar
\sphinxstyleliteralstrong{\sphinxupquote{ValueError}} \textendash{} if we have no solutions in the determined interval

\end{itemize}

\sphinxlineitem{Returns}
\sphinxAtStartPar
the reject angle xi(t)

\sphinxlineitem{Return type}
\sphinxAtStartPar
double

\end{description}\end{quote}

\end{fulllineitems}

\index{retire\_contrib\_aller() (in module analytical\_reject)@\spxentry{retire\_contrib\_aller()}\spxextra{in module analytical\_reject}}

\begin{fulllineitems}
\phantomsection\label{\detokenize{analytical_reject:analytical_reject.retire_contrib_aller}}
\pysigstartsignatures
\pysiglinewithargsret
{\sphinxcode{\sphinxupquote{analytical\_reject.}}\sphinxbfcode{\sphinxupquote{retire\_contrib\_aller}}}
{\sphinxparam{\DUrole{n}{gamma}}\sphinxparamcomma \sphinxparam{\DUrole{n}{plaquette}}\sphinxparamcomma \sphinxparam{\DUrole{n}{index\_lambda}}}
{}
\pysigstopsignatures
\sphinxAtStartPar
Removes a forward contribution from gamma.
\begin{quote}\begin{description}
\sphinxlineitem{Parameters}\begin{itemize}
\item {} 
\sphinxAtStartPar
\sphinxstyleliteralstrong{\sphinxupquote{gamma}} (\sphinxstyleliteralemphasis{\sphinxupquote{double}}) \textendash{} the real number to reduce

\item {} 
\sphinxAtStartPar
\sphinxstyleliteralstrong{\sphinxupquote{plaquette}} (\sphinxstyleliteralemphasis{\sphinxupquote{numpy.array}}) \textendash{} the plaquette used to compute a,b coefficients

\item {} 
\sphinxAtStartPar
\sphinxstyleliteralstrong{\sphinxupquote{index\_lambda}} (\sphinxstyleliteralemphasis{\sphinxupquote{int}}) \textendash{} the index of the gell\sphinxhyphen{}mann matrix, defines the contribution to remove

\end{itemize}

\sphinxlineitem{Returns}
\sphinxAtStartPar
reduced gamma

\sphinxlineitem{Return type}
\sphinxAtStartPar
double

\end{description}\end{quote}

\end{fulllineitems}

\index{signes() (in module analytical\_reject)@\spxentry{signes()}\spxextra{in module analytical\_reject}}

\begin{fulllineitems}
\phantomsection\label{\detokenize{analytical_reject:analytical_reject.signes}}
\pysigstartsignatures
\pysiglinewithargsret
{\sphinxcode{\sphinxupquote{analytical\_reject.}}\sphinxbfcode{\sphinxupquote{signes}}}
{\sphinxparam{\DUrole{n}{a}}\sphinxparamcomma \sphinxparam{\DUrole{n}{b}}}
{}
\pysigstopsignatures
\sphinxAtStartPar
Returns a tuple of either 1, \sphinxhyphen{}1, or 0 based on the signs of the real numbers a and b

\sphinxAtStartPar
Args:
a (double): real number
b (double): real number
\begin{quote}\begin{description}
\sphinxlineitem{Returns}
\sphinxAtStartPar
returns (1, \_) if a \textgreater{} 0, (\sphinxhyphen{}1, \_) if a \textless{} 0, (0, \_) if a = 0; same for b

\sphinxlineitem{Return type}
\sphinxAtStartPar
(int, int)

\end{description}\end{quote}

\end{fulllineitems}

\index{solve\_trig\_eq() (in module analytical\_reject)@\spxentry{solve\_trig\_eq()}\spxextra{in module analytical\_reject}}

\begin{fulllineitems}
\phantomsection\label{\detokenize{analytical_reject:analytical_reject.solve_trig_eq}}
\pysigstartsignatures
\pysiglinewithargsret
{\sphinxcode{\sphinxupquote{analytical\_reject.}}\sphinxbfcode{\sphinxupquote{solve\_trig\_eq}}}
{\sphinxparam{\DUrole{n}{a}}\sphinxparamcomma \sphinxparam{\DUrole{n}{b}}\sphinxparamcomma \sphinxparam{\DUrole{n}{c}}\sphinxparamcomma \sphinxparam{\DUrole{n}{gamma}}}
{}
\pysigstopsignatures
\sphinxAtStartPar
Solves the equation acos(x)+bsin(x)+c = gamma modulo 2pi.
\begin{quote}\begin{description}
\sphinxlineitem{Parameters}\begin{itemize}
\item {} 
\sphinxAtStartPar
\sphinxstyleliteralstrong{\sphinxupquote{a}} (\sphinxstyleliteralemphasis{\sphinxupquote{double}}) \textendash{} coefficient of cos

\item {} 
\sphinxAtStartPar
\sphinxstyleliteralstrong{\sphinxupquote{b}} (\sphinxstyleliteralemphasis{\sphinxupquote{double}}) \textendash{} coefficient of sin

\item {} 
\sphinxAtStartPar
\sphinxstyleliteralstrong{\sphinxupquote{c}} (\sphinxstyleliteralemphasis{\sphinxupquote{double}}) \textendash{} constant coefficient

\item {} 
\sphinxAtStartPar
\sphinxstyleliteralstrong{\sphinxupquote{gamma}} (\sphinxstyleliteralemphasis{\sphinxupquote{double}}) \textendash{} rhs of the equation

\end{itemize}

\sphinxlineitem{Raises}
\sphinxAtStartPar
\sphinxstyleliteralstrong{\sphinxupquote{ValueError}} \textendash{} if gamma\sphinxhyphen{}c must be \textgreater{} sqrt(a\textasciicircum{}2 + b\textasciicircum{}2), no solution possible

\sphinxlineitem{Returns}
\sphinxAtStartPar
the 2 solutions ordered

\sphinxlineitem{Return type}
\sphinxAtStartPar
double

\end{description}\end{quote}

\end{fulllineitems}


\sphinxstepscope


\section{numerical\_reject module}
\label{\detokenize{numerical_reject:module-numerical_reject}}\label{\detokenize{numerical_reject:numerical-reject-module}}\label{\detokenize{numerical_reject::doc}}\index{module@\spxentry{module}!numerical\_reject@\spxentry{numerical\_reject}}\index{numerical\_reject@\spxentry{numerical\_reject}!module@\spxentry{module}}
\sphinxAtStartPar
Generation of rejects by numerical resolution of the inversion equation. Available only for lambda\_8.
\index{coeffs\_8() (in module numerical\_reject)@\spxentry{coeffs\_8()}\spxextra{in module numerical\_reject}}

\begin{fulllineitems}
\phantomsection\label{\detokenize{numerical_reject:numerical_reject.coeffs_8}}
\pysigstartsignatures
\pysiglinewithargsret
{\sphinxcode{\sphinxupquote{numerical\_reject.}}\sphinxbfcode{\sphinxupquote{coeffs\_8}}}
{\sphinxparam{\DUrole{n}{plaquette}}}
{}
\pysigstopsignatures
\sphinxAtStartPar
Give the coefficients of the trigonometric function for lambda\_8 (cf pdf)
\begin{quote}\begin{description}
\sphinxlineitem{Parameters}
\sphinxAtStartPar
\sphinxstyleliteralstrong{\sphinxupquote{plaquette}} (\sphinxstyleliteralemphasis{\sphinxupquote{numpy.array}}) \textendash{} 3x3 SU(3) matrix

\sphinxlineitem{Returns}
\sphinxAtStartPar
a,b,c,d coefficients (cf pdf)

\sphinxlineitem{Return type}
\sphinxAtStartPar
(double, double, double, double)

\end{description}\end{quote}

\end{fulllineitems}

\index{coeffs\_quartic() (in module numerical\_reject)@\spxentry{coeffs\_quartic()}\spxextra{in module numerical\_reject}}

\begin{fulllineitems}
\phantomsection\label{\detokenize{numerical_reject:numerical_reject.coeffs_quartic}}
\pysigstartsignatures
\pysiglinewithargsret
{\sphinxcode{\sphinxupquote{numerical\_reject.}}\sphinxbfcode{\sphinxupquote{coeffs\_quartic}}}
{\sphinxparam{\DUrole{n}{plaquette}}}
{}
\pysigstopsignatures
\sphinxAtStartPar
“Return the coefficients of the numerator of the quartic equation for the derivative of the energy, with the dominant coefficient first, obtained using the change of variable t = tan(xi/(2sqrt(3))).”
\begin{quote}\begin{description}
\sphinxlineitem{Parameters}
\sphinxAtStartPar
\sphinxstyleliteralstrong{\sphinxupquote{plaquette}} (\sphinxstyleliteralemphasis{\sphinxupquote{numpy.array}}) \textendash{} 3x3 SU(3) matrix

\sphinxlineitem{Returns}
\sphinxAtStartPar
list of the polynomial coefficients

\sphinxlineitem{Return type}
\sphinxAtStartPar
list

\end{description}\end{quote}

\end{fulllineitems}

\index{contrib() (in module numerical\_reject)@\spxentry{contrib()}\spxextra{in module numerical\_reject}}

\begin{fulllineitems}
\phantomsection\label{\detokenize{numerical_reject:numerical_reject.contrib}}
\pysigstartsignatures
\pysiglinewithargsret
{\sphinxcode{\sphinxupquote{numerical\_reject.}}\sphinxbfcode{\sphinxupquote{contrib}}}
{\sphinxparam{\DUrole{n}{plaquette}}\sphinxparamcomma \sphinxparam{\DUrole{n}{a}}\sphinxparamcomma \sphinxparam{\DUrole{n}{b}}}
{}
\pysigstopsignatures
\sphinxAtStartPar
Returns the value of the integral of max(0, E’+) between a and b.
\begin{quote}\begin{description}
\sphinxlineitem{Parameters}\begin{itemize}
\item {} 
\sphinxAtStartPar
\sphinxstyleliteralstrong{\sphinxupquote{plaquette}} (\sphinxstyleliteralemphasis{\sphinxupquote{numpy.array}}) \textendash{} 3x3 SU(3) matrix

\item {} 
\sphinxAtStartPar
\sphinxstyleliteralstrong{\sphinxupquote{a}} (\sphinxstyleliteralemphasis{\sphinxupquote{double}}) \textendash{} lower bound

\item {} 
\sphinxAtStartPar
\sphinxstyleliteralstrong{\sphinxupquote{b}} (\sphinxstyleliteralemphasis{\sphinxupquote{double}}) \textendash{} upper bound

\end{itemize}

\sphinxlineitem{Raises}
\sphinxAtStartPar
\sphinxstyleliteralstrong{\sphinxupquote{ValueError}} \textendash{} a and b must be in 0,4pi

\sphinxlineitem{Returns}
\sphinxAtStartPar
value of the integral

\sphinxlineitem{Return type}
\sphinxAtStartPar
double

\end{description}\end{quote}

\end{fulllineitems}

\index{derivee\_f\_lambda\_8() (in module numerical\_reject)@\spxentry{derivee\_f\_lambda\_8()}\spxextra{in module numerical\_reject}}

\begin{fulllineitems}
\phantomsection\label{\detokenize{numerical_reject:numerical_reject.derivee_f_lambda_8}}
\pysigstartsignatures
\pysiglinewithargsret
{\sphinxcode{\sphinxupquote{numerical\_reject.}}\sphinxbfcode{\sphinxupquote{derivee\_f\_lambda\_8}}}
{\sphinxparam{\DUrole{n}{xi}}\sphinxparamcomma \sphinxparam{\DUrole{n}{a}}\sphinxparamcomma \sphinxparam{\DUrole{n}{b}}\sphinxparamcomma \sphinxparam{\DUrole{n}{c}}\sphinxparamcomma \sphinxparam{\DUrole{n}{d}}}
{}
\pysigstopsignatures
\sphinxAtStartPar
The trigonometric function of the derivative of the energy for lambda\_8.
\begin{quote}\begin{description}
\sphinxlineitem{Parameters}\begin{itemize}
\item {} 
\sphinxAtStartPar
\sphinxstyleliteralstrong{\sphinxupquote{xi}} (\sphinxstyleliteralemphasis{\sphinxupquote{double}}) \textendash{} angle

\item {} 
\sphinxAtStartPar
\sphinxstyleliteralstrong{\sphinxupquote{a}} (\sphinxstyleliteralemphasis{\sphinxupquote{double}}) \textendash{} coefficient of the trig function

\item {} 
\sphinxAtStartPar
\sphinxstyleliteralstrong{\sphinxupquote{b}} (\sphinxstyleliteralemphasis{\sphinxupquote{double}}) \textendash{} coefficient of the trig function

\item {} 
\sphinxAtStartPar
\sphinxstyleliteralstrong{\sphinxupquote{c}} (\sphinxstyleliteralemphasis{\sphinxupquote{double}}) \textendash{} coefficient of the trig function

\item {} 
\sphinxAtStartPar
\sphinxstyleliteralstrong{\sphinxupquote{d}} (\sphinxstyleliteralemphasis{\sphinxupquote{double}}) \textendash{} coefficient of the trig function

\end{itemize}

\sphinxlineitem{Returns}
\sphinxAtStartPar
derivative of the energy for lambda\_8

\sphinxlineitem{Return type}
\sphinxAtStartPar
double

\end{description}\end{quote}

\end{fulllineitems}

\index{derivee\_f\_lambda\_8\_plus() (in module numerical\_reject)@\spxentry{derivee\_f\_lambda\_8\_plus()}\spxextra{in module numerical\_reject}}

\begin{fulllineitems}
\phantomsection\label{\detokenize{numerical_reject:numerical_reject.derivee_f_lambda_8_plus}}
\pysigstartsignatures
\pysiglinewithargsret
{\sphinxcode{\sphinxupquote{numerical\_reject.}}\sphinxbfcode{\sphinxupquote{derivee\_f\_lambda\_8\_plus}}}
{\sphinxparam{\DUrole{n}{t}}\sphinxparamcomma \sphinxparam{\DUrole{n}{plaquette}}}
{}
\pysigstopsignatures
\sphinxAtStartPar
Computes the max between 0 and the derivative of the energy for lambda\_8 with parametrization xi(t) and t in 0,4pi
\begin{quote}\begin{description}
\sphinxlineitem{Parameters}\begin{itemize}
\item {} 
\sphinxAtStartPar
\sphinxstyleliteralstrong{\sphinxupquote{t}} (\sphinxstyleliteralemphasis{\sphinxupquote{double}}) \textendash{} parameter for xi

\item {} 
\sphinxAtStartPar
\sphinxstyleliteralstrong{\sphinxupquote{plaquette}} (\sphinxstyleliteralemphasis{\sphinxupquote{numpy.array}}) \textendash{} 3x3 SU(3) matrix

\end{itemize}

\sphinxlineitem{Returns}
\sphinxAtStartPar
value of the function

\sphinxlineitem{Return type}
\sphinxAtStartPar
double

\end{description}\end{quote}

\end{fulllineitems}

\index{derivee\_f\_lambda\_8\_plus\_aller() (in module numerical\_reject)@\spxentry{derivee\_f\_lambda\_8\_plus\_aller()}\spxextra{in module numerical\_reject}}

\begin{fulllineitems}
\phantomsection\label{\detokenize{numerical_reject:numerical_reject.derivee_f_lambda_8_plus_aller}}
\pysigstartsignatures
\pysiglinewithargsret
{\sphinxcode{\sphinxupquote{numerical\_reject.}}\sphinxbfcode{\sphinxupquote{derivee\_f\_lambda\_8\_plus\_aller}}}
{\sphinxparam{\DUrole{n}{t}}\sphinxparamcomma \sphinxparam{\DUrole{n}{plaquette}}}
{}
\pysigstopsignatures
\sphinxAtStartPar
Computes the max between 0 and the derivative of the energy for lambda\_8 when xi(t) is moving foward 0 \sphinxhyphen{}\textgreater{} 2pi
\begin{quote}\begin{description}
\sphinxlineitem{Parameters}\begin{itemize}
\item {} 
\sphinxAtStartPar
\sphinxstyleliteralstrong{\sphinxupquote{t}} (\sphinxstyleliteralemphasis{\sphinxupquote{double}}) \textendash{} angle

\item {} 
\sphinxAtStartPar
\sphinxstyleliteralstrong{\sphinxupquote{plaquette}} (\sphinxstyleliteralemphasis{\sphinxupquote{numpy.array}}) \textendash{} 3x3 SU(3) matrix

\end{itemize}

\sphinxlineitem{Raises}
\sphinxAtStartPar
\sphinxstyleliteralstrong{\sphinxupquote{ValueError}} \textendash{} t must be in {[}0,2pi{]}

\sphinxlineitem{Returns}
\sphinxAtStartPar
value of the function

\sphinxlineitem{Return type}
\sphinxAtStartPar
double

\end{description}\end{quote}

\end{fulllineitems}

\index{derivee\_f\_lambda\_8\_plus\_retour() (in module numerical\_reject)@\spxentry{derivee\_f\_lambda\_8\_plus\_retour()}\spxextra{in module numerical\_reject}}

\begin{fulllineitems}
\phantomsection\label{\detokenize{numerical_reject:numerical_reject.derivee_f_lambda_8_plus_retour}}
\pysigstartsignatures
\pysiglinewithargsret
{\sphinxcode{\sphinxupquote{numerical\_reject.}}\sphinxbfcode{\sphinxupquote{derivee\_f\_lambda\_8\_plus\_retour}}}
{\sphinxparam{\DUrole{n}{t}}\sphinxparamcomma \sphinxparam{\DUrole{n}{plaquette}}}
{}
\pysigstopsignatures
\sphinxAtStartPar
Computes the max between 0 and the derivative of the energy for lambda\_8 when xi(t) is moving backward 0 \textless{}\sphinxhyphen{} 2pi
\begin{quote}\begin{description}
\sphinxlineitem{Parameters}\begin{itemize}
\item {} 
\sphinxAtStartPar
\sphinxstyleliteralstrong{\sphinxupquote{t}} (\sphinxstyleliteralemphasis{\sphinxupquote{double}}) \textendash{} backward parametrization of xi (cf pdf xi(t)  = 4pi \sphinxhyphen{} t)

\item {} 
\sphinxAtStartPar
\sphinxstyleliteralstrong{\sphinxupquote{plaquette}} (\sphinxstyleliteralemphasis{\sphinxupquote{numpy.array}}) \textendash{} 3x3 SU(3) matrix

\end{itemize}

\sphinxlineitem{Raises}
\sphinxAtStartPar
\sphinxstyleliteralstrong{\sphinxupquote{ValueError}} \textendash{} t must be in {[}2pi,4pi{]}

\sphinxlineitem{Returns}
\sphinxAtStartPar
value of the function

\sphinxlineitem{Return type}
\sphinxAtStartPar
double

\end{description}\end{quote}

\end{fulllineitems}

\index{f\_8() (in module numerical\_reject)@\spxentry{f\_8()}\spxextra{in module numerical\_reject}}

\begin{fulllineitems}
\phantomsection\label{\detokenize{numerical_reject:numerical_reject.f_8}}
\pysigstartsignatures
\pysiglinewithargsret
{\sphinxcode{\sphinxupquote{numerical\_reject.}}\sphinxbfcode{\sphinxupquote{f\_8}}}
{\sphinxparam{\DUrole{n}{x}}\sphinxparamcomma \sphinxparam{\DUrole{n}{plaquette}}\sphinxparamcomma \sphinxparam{\DUrole{n}{a}}\sphinxparamcomma \sphinxparam{\DUrole{n}{gamma}}}
{}
\pysigstopsignatures
\sphinxAtStartPar
The function whose roots are the solutions of the inversion equation for the rejects
\begin{quote}\begin{description}
\sphinxlineitem{Parameters}\begin{itemize}
\item {} 
\sphinxAtStartPar
\sphinxstyleliteralstrong{\sphinxupquote{x}} (\sphinxstyleliteralemphasis{\sphinxupquote{double}}) \textendash{} real number

\item {} 
\sphinxAtStartPar
\sphinxstyleliteralstrong{\sphinxupquote{plaquette}} (\sphinxstyleliteralemphasis{\sphinxupquote{numpy.array}}) \textendash{} 3x3 SU(3) matrix

\item {} 
\sphinxAtStartPar
\sphinxstyleliteralstrong{\sphinxupquote{a}} (\sphinxstyleliteralemphasis{\sphinxupquote{double}}) \textendash{} lower bound of the integral

\item {} 
\sphinxAtStartPar
\sphinxstyleliteralstrong{\sphinxupquote{gamma}} (\sphinxstyleliteralemphasis{\sphinxupquote{double}}) \textendash{} real number

\end{itemize}

\sphinxlineitem{Returns}
\sphinxAtStartPar
integral of max(0, derivative) \sphinxhyphen{} gamma

\sphinxlineitem{Return type}
\sphinxAtStartPar
double

\end{description}\end{quote}

\end{fulllineitems}

\index{intervalle\_signes() (in module numerical\_reject)@\spxentry{intervalle\_signes()}\spxextra{in module numerical\_reject}}

\begin{fulllineitems}
\phantomsection\label{\detokenize{numerical_reject:numerical_reject.intervalle_signes}}
\pysigstartsignatures
\pysiglinewithargsret
{\sphinxcode{\sphinxupquote{numerical\_reject.}}\sphinxbfcode{\sphinxupquote{intervalle\_signes}}}
{\sphinxparam{\DUrole{n}{plaquette}}}
{}
\pysigstopsignatures
\sphinxAtStartPar
Returns the lists of intervals where the derivative of the energy is positive and negative (useful for the return) on 0 to 2pi
\begin{quote}\begin{description}
\sphinxlineitem{Parameters}
\sphinxAtStartPar
\sphinxstyleliteralstrong{\sphinxupquote{plaquette}} (\sphinxstyleliteralemphasis{\sphinxupquote{numpy.array}}) \textendash{} 3x3 SU(3) matrix

\sphinxlineitem{Returns}
\sphinxAtStartPar
lists of intervals where the derivative of the energy is positive and negative

\sphinxlineitem{Return type}
\sphinxAtStartPar
list of tuple, list of tuples

\end{description}\end{quote}

\end{fulllineitems}

\index{intervalles\_aller\_retour() (in module numerical\_reject)@\spxentry{intervalles\_aller\_retour()}\spxextra{in module numerical\_reject}}

\begin{fulllineitems}
\phantomsection\label{\detokenize{numerical_reject:numerical_reject.intervalles_aller_retour}}
\pysigstartsignatures
\pysiglinewithargsret
{\sphinxcode{\sphinxupquote{numerical\_reject.}}\sphinxbfcode{\sphinxupquote{intervalles\_aller\_retour}}}
{\sphinxparam{\DUrole{n}{plaquette}}}
{}
\pysigstopsignatures
\sphinxAtStartPar
Returns the positive intervals for t for forward and return paths
\begin{quote}\begin{description}
\sphinxlineitem{Parameters}
\sphinxAtStartPar
\sphinxstyleliteralstrong{\sphinxupquote{plaquette}} (\sphinxstyleliteralemphasis{\sphinxupquote{numpy.array}}) \textendash{} 3x3 SU(3) matrix

\sphinxlineitem{Returns}
\sphinxAtStartPar
positive intervals for the foward path, positive intervals for the backwards path (for variable t)

\sphinxlineitem{Return type}
\sphinxAtStartPar
list of tuples, list of tuples

\end{description}\end{quote}

\end{fulllineitems}

\index{real\_roots() (in module numerical\_reject)@\spxentry{real\_roots()}\spxextra{in module numerical\_reject}}

\begin{fulllineitems}
\phantomsection\label{\detokenize{numerical_reject:numerical_reject.real_roots}}
\pysigstartsignatures
\pysiglinewithargsret
{\sphinxcode{\sphinxupquote{numerical\_reject.}}\sphinxbfcode{\sphinxupquote{real\_roots}}}
{\sphinxparam{\DUrole{n}{plaquette}}}
{}
\pysigstopsignatures
\sphinxAtStartPar
Return the real roots of the numerator of the quartic equation corresponding to the derivative of the energy. The function root calculates the eigenvalues of the companion matrix associated with the polynomial.
\begin{quote}\begin{description}
\sphinxlineitem{Parameters}
\sphinxAtStartPar
\sphinxstyleliteralstrong{\sphinxupquote{plaquette}} (\sphinxstyleliteralemphasis{\sphinxupquote{numpy.array}}) \textendash{} 3x3 SU(3) matrix

\sphinxlineitem{Returns}
\sphinxAtStartPar
list of the real roots

\sphinxlineitem{Return type}
\sphinxAtStartPar
list

\end{description}\end{quote}

\end{fulllineitems}

\index{reduce\_gamma\_8() (in module numerical\_reject)@\spxentry{reduce\_gamma\_8()}\spxextra{in module numerical\_reject}}

\begin{fulllineitems}
\phantomsection\label{\detokenize{numerical_reject:numerical_reject.reduce_gamma_8}}
\pysigstartsignatures
\pysiglinewithargsret
{\sphinxcode{\sphinxupquote{numerical\_reject.}}\sphinxbfcode{\sphinxupquote{reduce\_gamma\_8}}}
{\sphinxparam{\DUrole{n}{gamma}}\sphinxparamcomma \sphinxparam{\DUrole{n}{plaquette}}}
{}
\pysigstopsignatures
\sphinxAtStartPar
Takes a gamma, reduces it to its value in the contribution of the interval containing the solution, and determines the interval in which to solve the equation.
\begin{quote}\begin{description}
\sphinxlineitem{Parameters}\begin{itemize}
\item {} 
\sphinxAtStartPar
\sphinxstyleliteralstrong{\sphinxupquote{gamma}} (\sphinxstyleliteralemphasis{\sphinxupquote{double}}) \textendash{} real number

\item {} 
\sphinxAtStartPar
\sphinxstyleliteralstrong{\sphinxupquote{plaquette}} (\sphinxstyleliteralemphasis{\sphinxupquote{numpy.array}}) \textendash{} 3x3 SU(3) matrix

\end{itemize}

\sphinxlineitem{Raises}
\sphinxAtStartPar
\sphinxstyleliteralstrong{\sphinxupquote{ValueError}} \textendash{} if the computed bounds of the interval are in the wrong order

\sphinxlineitem{Returns}
\sphinxAtStartPar
reduced gamma, lower bound, upper bound

\sphinxlineitem{Return type}
\sphinxAtStartPar
double, double, double

\end{description}\end{quote}

\end{fulllineitems}

\index{roots\_xi() (in module numerical\_reject)@\spxentry{roots\_xi()}\spxextra{in module numerical\_reject}}

\begin{fulllineitems}
\phantomsection\label{\detokenize{numerical_reject:numerical_reject.roots_xi}}
\pysigstartsignatures
\pysiglinewithargsret
{\sphinxcode{\sphinxupquote{numerical\_reject.}}\sphinxbfcode{\sphinxupquote{roots\_xi}}}
{\sphinxparam{\DUrole{n}{plaquette}}}
{}
\pysigstopsignatures
\sphinxAtStartPar
Return the roots of the derivative of the energy for xi ranging from {[}0, 2pi{]}
\begin{quote}\begin{description}
\sphinxlineitem{Parameters}
\sphinxAtStartPar
\sphinxstyleliteralstrong{\sphinxupquote{plaquette}} (\sphinxstyleliteralemphasis{\sphinxupquote{numpy.array}}) \textendash{} 3x3 SU(3) matrix

\sphinxlineitem{Returns}
\sphinxAtStartPar
list of roots of the derivative of the energy for lambda\_8, variable xi

\sphinxlineitem{Return type}
\sphinxAtStartPar
list

\end{description}\end{quote}

\end{fulllineitems}

\index{signe\_entre\_racines() (in module numerical\_reject)@\spxentry{signe\_entre\_racines()}\spxextra{in module numerical\_reject}}

\begin{fulllineitems}
\phantomsection\label{\detokenize{numerical_reject:numerical_reject.signe_entre_racines}}
\pysigstartsignatures
\pysiglinewithargsret
{\sphinxcode{\sphinxupquote{numerical\_reject.}}\sphinxbfcode{\sphinxupquote{signe\_entre\_racines}}}
{\sphinxparam{\DUrole{n}{plaquette}}}
{}
\pysigstopsignatures
\sphinxAtStartPar
Returns a vector of \sphinxhyphen{}1 and 1 depending on the sign of the derivative of the energy between the roots, only for xi between 0 and 2pi (forward crossing).
\begin{quote}\begin{description}
\sphinxlineitem{Parameters}
\sphinxAtStartPar
\sphinxstyleliteralstrong{\sphinxupquote{plaquette}} (\sphinxstyleliteralemphasis{\sphinxupquote{numpy.array}}) \textendash{} 3x3 SU(3) matrix

\sphinxlineitem{Returns}
\sphinxAtStartPar
list of 1 or \sphinxhyphen{}1 depending on the sign of the function between the roots

\sphinxlineitem{Return type}
\sphinxAtStartPar
list of int

\end{description}\end{quote}

\end{fulllineitems}

\index{solve\_8() (in module numerical\_reject)@\spxentry{solve\_8()}\spxextra{in module numerical\_reject}}

\begin{fulllineitems}
\phantomsection\label{\detokenize{numerical_reject:numerical_reject.solve_8}}
\pysigstartsignatures
\pysiglinewithargsret
{\sphinxcode{\sphinxupquote{numerical\_reject.}}\sphinxbfcode{\sphinxupquote{solve\_8}}}
{\sphinxparam{\DUrole{n}{gamma}}\sphinxparamcomma \sphinxparam{\DUrole{n}{plaquette}}}
{}
\pysigstopsignatures
\sphinxAtStartPar
Solves the inversion equation for lambda\_8 for a given number on the rhs of the inversion equation and a given plaquette
\begin{quote}\begin{description}
\sphinxlineitem{Parameters}\begin{itemize}
\item {} 
\sphinxAtStartPar
\sphinxstyleliteralstrong{\sphinxupquote{gamma}} (\sphinxstyleliteralemphasis{\sphinxupquote{double}}) \textendash{} the rhs of the inversion equation

\item {} 
\sphinxAtStartPar
\sphinxstyleliteralstrong{\sphinxupquote{plaquette}} (\sphinxstyleliteralemphasis{\sphinxupquote{numpy.array}}) \textendash{} 3x3 SU(3) matrix

\end{itemize}

\sphinxlineitem{Raises}
\sphinxAtStartPar
\sphinxstyleliteralstrong{\sphinxupquote{ValueError}} \textendash{} the numerically computed reject angle is not right (we don’t retrieve gamma)

\sphinxlineitem{Returns}
\sphinxAtStartPar
solution of the inversion equation

\sphinxlineitem{Return type}
\sphinxAtStartPar
double

\end{description}\end{quote}

\end{fulllineitems}


\sphinxstepscope


\section{ECMC module}
\label{\detokenize{ECMC:module-ECMC}}\label{\detokenize{ECMC:ecmc-module}}\label{\detokenize{ECMC::doc}}\index{module@\spxentry{module}!ECMC@\spxentry{ECMC}}\index{ECMC@\spxentry{ECMC}!module@\spxentry{module}}\index{ECMC\_samples() (in module ECMC)@\spxentry{ECMC\_samples()}\spxextra{in module ECMC}}

\begin{fulllineitems}
\phantomsection\label{\detokenize{ECMC:ECMC.ECMC_samples}}
\pysigstartsignatures
\pysiglinewithargsret
{\sphinxcode{\sphinxupquote{ECMC.}}\sphinxbfcode{\sphinxupquote{ECMC\_samples}}}
{\sphinxparam{\DUrole{n}{L}\DUrole{o}{=}\DUrole{default_value}{4}}\sphinxparamcomma \sphinxparam{\DUrole{n}{T}\DUrole{o}{=}\DUrole{default_value}{4}}\sphinxparamcomma \sphinxparam{\DUrole{n}{cold}\DUrole{o}{=}\DUrole{default_value}{True}}\sphinxparamcomma \sphinxparam{\DUrole{n}{beta}\DUrole{o}{=}\DUrole{default_value}{2.55}}\sphinxparamcomma \sphinxparam{\DUrole{n}{angle\_l}\DUrole{o}{=}\DUrole{default_value}{120}}\sphinxparamcomma \sphinxparam{\DUrole{n}{param\_lambda\_l}\DUrole{o}{=}\DUrole{default_value}{15}}\sphinxparamcomma \sphinxparam{\DUrole{n}{param\_pos\_l}\DUrole{o}{=}\DUrole{default_value}{30}}\sphinxparamcomma \sphinxparam{\DUrole{n}{n}\DUrole{o}{=}\DUrole{default_value}{5}}}
{}
\pysigstopsignatures
\sphinxAtStartPar
Returns n samples of gauge configurations sampled from the probability distribution induced by the Wilson action
\begin{quote}\begin{description}
\sphinxlineitem{Parameters}\begin{itemize}
\item {} 
\sphinxAtStartPar
\sphinxstyleliteralstrong{\sphinxupquote{L}} (\sphinxstyleliteralemphasis{\sphinxupquote{int}}\sphinxstyleliteralemphasis{\sphinxupquote{, }}\sphinxstyleliteralemphasis{\sphinxupquote{optional}}) \textendash{} spatial length of the lattice. Defaults to 4.

\item {} 
\sphinxAtStartPar
\sphinxstyleliteralstrong{\sphinxupquote{T}} (\sphinxstyleliteralemphasis{\sphinxupquote{int}}\sphinxstyleliteralemphasis{\sphinxupquote{, }}\sphinxstyleliteralemphasis{\sphinxupquote{optional}}) \textendash{} temporal length of the lattice. Defaults to 4.

\item {} 
\sphinxAtStartPar
\sphinxstyleliteralstrong{\sphinxupquote{cold}} (\sphinxstyleliteralemphasis{\sphinxupquote{bool}}\sphinxstyleliteralemphasis{\sphinxupquote{, }}\sphinxstyleliteralemphasis{\sphinxupquote{optional}}) \textendash{} True if cold start, False for hot start. Defaults to True.

\item {} 
\sphinxAtStartPar
\sphinxstyleliteralstrong{\sphinxupquote{beta}} (\sphinxstyleliteralemphasis{\sphinxupquote{double}}) \textendash{} Wilson action inverse coupling constant

\item {} 
\sphinxAtStartPar
\sphinxstyleliteralstrong{\sphinxupquote{angle\_l}} (\sphinxstyleliteralemphasis{\sphinxupquote{double}}) \textendash{} total distance upon which we select a sample

\item {} 
\sphinxAtStartPar
\sphinxstyleliteralstrong{\sphinxupquote{param\_lambda\_l}} (\sphinxstyleliteralemphasis{\sphinxupquote{double}}) \textendash{} poisson law parameter of the total distance upon which we change the direction of updating

\item {} 
\sphinxAtStartPar
\sphinxstyleliteralstrong{\sphinxupquote{param\_pos\_l}} (\sphinxstyleliteralemphasis{\sphinxupquote{double}}) \textendash{} poisson law parameter of the total distance upon which we randomly change the updated link

\item {} 
\sphinxAtStartPar
\sphinxstyleliteralstrong{\sphinxupquote{n}} (\sphinxstyleliteralemphasis{\sphinxupquote{int}}\sphinxstyleliteralemphasis{\sphinxupquote{, }}\sphinxstyleliteralemphasis{\sphinxupquote{optional}}) \textendash{} number of samples to generate. Defaults to 5.

\end{itemize}

\sphinxlineitem{Returns}
\sphinxAtStartPar
list of n samples of gauge configurations

\sphinxlineitem{Return type}
\sphinxAtStartPar
list of numpy.array

\end{description}\end{quote}

\end{fulllineitems}

\index{ECMC\_step() (in module ECMC)@\spxentry{ECMC\_step()}\spxextra{in module ECMC}}

\begin{fulllineitems}
\phantomsection\label{\detokenize{ECMC:ECMC.ECMC_step}}
\pysigstartsignatures
\pysiglinewithargsret
{\sphinxcode{\sphinxupquote{ECMC.}}\sphinxbfcode{\sphinxupquote{ECMC\_step}}}
{\sphinxparam{\DUrole{n}{conf}}\sphinxparamcomma \sphinxparam{\DUrole{n}{x}}\sphinxparamcomma \sphinxparam{\DUrole{n}{t}}\sphinxparamcomma \sphinxparam{\DUrole{n}{mu}}\sphinxparamcomma \sphinxparam{\DUrole{n}{index\_lambda}}\sphinxparamcomma \sphinxparam{\DUrole{n}{beta}}}
{}
\pysigstopsignatures
\sphinxAtStartPar
“Performs an ECMC step: returns the matrix X = exp(i xi lambda\_\{indice\_lambda\}) and the angle xi of the rejection event for the link (x,t,mu) of the configuration conf, as well as next\_link, the link in the plaquette responsible for the rejection that will need to be changed afterward. index\_lambda can only take values 2, 3, 5, or 8.
\begin{quote}\begin{description}
\sphinxlineitem{Parameters}\begin{itemize}
\item {} 
\sphinxAtStartPar
\sphinxstyleliteralstrong{\sphinxupquote{conf}} (\sphinxstyleliteralemphasis{\sphinxupquote{numpy.array}}) \textendash{} 1+1d lattice gauge configuration

\item {} 
\sphinxAtStartPar
\sphinxstyleliteralstrong{\sphinxupquote{x}} (\sphinxstyleliteralemphasis{\sphinxupquote{int}}) \textendash{} spatial coordinate of the link

\item {} 
\sphinxAtStartPar
\sphinxstyleliteralstrong{\sphinxupquote{t}} (\sphinxstyleliteralemphasis{\sphinxupquote{int}}) \textendash{} temporal coordinate of the link

\item {} 
\sphinxAtStartPar
\sphinxstyleliteralstrong{\sphinxupquote{mu}} (\sphinxstyleliteralemphasis{\sphinxupquote{int}}) \textendash{} lorentz coordinate of the link

\item {} 
\sphinxAtStartPar
\sphinxstyleliteralstrong{\sphinxupquote{index\_lambda}} (\sphinxstyleliteralemphasis{\sphinxupquote{int}}) \textendash{} direction of updating

\item {} 
\sphinxAtStartPar
\sphinxstyleliteralstrong{\sphinxupquote{beta}} (\sphinxstyleliteralemphasis{\sphinxupquote{double}}) \textendash{} Wilson action inverse coupling constant

\end{itemize}

\sphinxlineitem{Raises}
\sphinxAtStartPar
\sphinxstyleliteralstrong{\sphinxupquote{ValueError}} \textendash{} index\_lambda must be 2,3,5 or 8

\sphinxlineitem{Returns}
\sphinxAtStartPar
updating matrix X, angle of updating xi, next link to update (x,t,mu)

\sphinxlineitem{Return type}
\sphinxAtStartPar
numpy.array, double, (int,int,int)

\end{description}\end{quote}

\end{fulllineitems}

\index{ECMC\_step\_l() (in module ECMC)@\spxentry{ECMC\_step\_l()}\spxextra{in module ECMC}}

\begin{fulllineitems}
\phantomsection\label{\detokenize{ECMC:ECMC.ECMC_step_l}}
\pysigstartsignatures
\pysiglinewithargsret
{\sphinxcode{\sphinxupquote{ECMC.}}\sphinxbfcode{\sphinxupquote{ECMC\_step\_l}}}
{\sphinxparam{\DUrole{n}{conf}}\sphinxparamcomma \sphinxparam{\DUrole{n}{x}}\sphinxparamcomma \sphinxparam{\DUrole{n}{t}}\sphinxparamcomma \sphinxparam{\DUrole{n}{mu}}\sphinxparamcomma \sphinxparam{\DUrole{n}{beta}}\sphinxparamcomma \sphinxparam{\DUrole{n}{angle\_l}}\sphinxparamcomma \sphinxparam{\DUrole{n}{param\_lambda\_l}}\sphinxparamcomma \sphinxparam{\DUrole{n}{param\_pos\_l}}}
{}
\pysigstopsignatures
\sphinxAtStartPar
Returns the configuration conf after ECMC steps starting from the point (x, t, mu) until the total angle traveled reaches angle\_l.
\begin{quote}\begin{description}
\sphinxlineitem{Parameters}\begin{itemize}
\item {} 
\sphinxAtStartPar
\sphinxstyleliteralstrong{\sphinxupquote{conf}} (\sphinxstyleliteralemphasis{\sphinxupquote{numpy.array}}) \textendash{} 1+1d lattice gauge configuration

\item {} 
\sphinxAtStartPar
\sphinxstyleliteralstrong{\sphinxupquote{x}} (\sphinxstyleliteralemphasis{\sphinxupquote{int}}) \textendash{} spatial coordinate of the link

\item {} 
\sphinxAtStartPar
\sphinxstyleliteralstrong{\sphinxupquote{t}} (\sphinxstyleliteralemphasis{\sphinxupquote{int}}) \textendash{} temporal coordinate of the link

\item {} 
\sphinxAtStartPar
\sphinxstyleliteralstrong{\sphinxupquote{mu}} (\sphinxstyleliteralemphasis{\sphinxupquote{int}}) \textendash{} lorentz coordinate of the link

\item {} 
\sphinxAtStartPar
\sphinxstyleliteralstrong{\sphinxupquote{index\_lambda}} (\sphinxstyleliteralemphasis{\sphinxupquote{int}}) \textendash{} direction of updating

\item {} 
\sphinxAtStartPar
\sphinxstyleliteralstrong{\sphinxupquote{beta}} (\sphinxstyleliteralemphasis{\sphinxupquote{double}}) \textendash{} Wilson action inverse coupling constant

\item {} 
\sphinxAtStartPar
\sphinxstyleliteralstrong{\sphinxupquote{angle\_l}} (\sphinxstyleliteralemphasis{\sphinxupquote{double}}) \textendash{} total distance upon which we select a sample

\item {} 
\sphinxAtStartPar
\sphinxstyleliteralstrong{\sphinxupquote{param\_lambda\_l}} (\sphinxstyleliteralemphasis{\sphinxupquote{double}}) \textendash{} poisson law parameter of the total distance upon which we change the direction of updating

\item {} 
\sphinxAtStartPar
\sphinxstyleliteralstrong{\sphinxupquote{param\_pos\_l}} (\sphinxstyleliteralemphasis{\sphinxupquote{double}}) \textendash{} poisson law parameter of the total distance upon which we randomly change the updated link

\end{itemize}

\sphinxlineitem{Raises}
\sphinxAtStartPar
\sphinxstyleliteralstrong{\sphinxupquote{ValueError}} \textendash{} angle\_l must be positive

\end{description}\end{quote}

\end{fulllineitems}

\index{random() (in module ECMC)@\spxentry{random()}\spxextra{in module ECMC}}

\begin{fulllineitems}
\phantomsection\label{\detokenize{ECMC:ECMC.random}}
\pysigstartsignatures
\pysiglinewithargsret
{\sphinxcode{\sphinxupquote{ECMC.}}\sphinxbfcode{\sphinxupquote{random}}}
{}
{}
\pysigstopsignatures
\end{fulllineitems}

\index{walker\_action() (in module ECMC)@\spxentry{walker\_action()}\spxextra{in module ECMC}}

\begin{fulllineitems}
\phantomsection\label{\detokenize{ECMC:ECMC.walker_action}}
\pysigstartsignatures
\pysiglinewithargsret
{\sphinxcode{\sphinxupquote{ECMC.}}\sphinxbfcode{\sphinxupquote{walker\_action}}}
{\sphinxparam{\DUrole{n}{L}\DUrole{o}{=}\DUrole{default_value}{4}}\sphinxparamcomma \sphinxparam{\DUrole{n}{T}\DUrole{o}{=}\DUrole{default_value}{4}}\sphinxparamcomma \sphinxparam{\DUrole{n}{cold}\DUrole{o}{=}\DUrole{default_value}{True}}\sphinxparamcomma \sphinxparam{\DUrole{n}{beta}\DUrole{o}{=}\DUrole{default_value}{2.55}}\sphinxparamcomma \sphinxparam{\DUrole{n}{angle\_l}\DUrole{o}{=}\DUrole{default_value}{120}}\sphinxparamcomma \sphinxparam{\DUrole{n}{param\_lambda\_l}\DUrole{o}{=}\DUrole{default_value}{15}}\sphinxparamcomma \sphinxparam{\DUrole{n}{param\_pos\_l}\DUrole{o}{=}\DUrole{default_value}{30}}\sphinxparamcomma \sphinxparam{\DUrole{n}{n}\DUrole{o}{=}\DUrole{default_value}{5}}}
{}
\pysigstopsignatures
\sphinxAtStartPar
Returns the list of actions of n samples of gauge configurations sampled from the probability distribution induced by the Wilson action
\begin{quote}\begin{description}
\sphinxlineitem{Parameters}\begin{itemize}
\item {} 
\sphinxAtStartPar
\sphinxstyleliteralstrong{\sphinxupquote{L}} (\sphinxstyleliteralemphasis{\sphinxupquote{int}}\sphinxstyleliteralemphasis{\sphinxupquote{, }}\sphinxstyleliteralemphasis{\sphinxupquote{optional}}) \textendash{} spatial length of the lattice. Defaults to 4.

\item {} 
\sphinxAtStartPar
\sphinxstyleliteralstrong{\sphinxupquote{T}} (\sphinxstyleliteralemphasis{\sphinxupquote{int}}\sphinxstyleliteralemphasis{\sphinxupquote{, }}\sphinxstyleliteralemphasis{\sphinxupquote{optional}}) \textendash{} temporal length of the lattice. Defaults to 4.

\item {} 
\sphinxAtStartPar
\sphinxstyleliteralstrong{\sphinxupquote{cold}} (\sphinxstyleliteralemphasis{\sphinxupquote{bool}}\sphinxstyleliteralemphasis{\sphinxupquote{, }}\sphinxstyleliteralemphasis{\sphinxupquote{optional}}) \textendash{} True if cold start, False for hot start. Defaults to True.

\item {} 
\sphinxAtStartPar
\sphinxstyleliteralstrong{\sphinxupquote{beta}} (\sphinxstyleliteralemphasis{\sphinxupquote{double}}) \textendash{} Wilson action inverse coupling constant

\item {} 
\sphinxAtStartPar
\sphinxstyleliteralstrong{\sphinxupquote{angle\_l}} (\sphinxstyleliteralemphasis{\sphinxupquote{double}}) \textendash{} total distance upon which we select a sample

\item {} 
\sphinxAtStartPar
\sphinxstyleliteralstrong{\sphinxupquote{param\_lambda\_l}} (\sphinxstyleliteralemphasis{\sphinxupquote{double}}) \textendash{} poisson law parameter of the total distance upon which we change the direction of updating

\item {} 
\sphinxAtStartPar
\sphinxstyleliteralstrong{\sphinxupquote{param\_pos\_l}} (\sphinxstyleliteralemphasis{\sphinxupquote{double}}) \textendash{} poisson law parameter of the total distance upon which we randomly change the updated link

\item {} 
\sphinxAtStartPar
\sphinxstyleliteralstrong{\sphinxupquote{n}} (\sphinxstyleliteralemphasis{\sphinxupquote{int}}\sphinxstyleliteralemphasis{\sphinxupquote{, }}\sphinxstyleliteralemphasis{\sphinxupquote{optional}}) \textendash{} number of samples to generate. Defaults to 5.

\end{itemize}

\sphinxlineitem{Returns}
\sphinxAtStartPar
list of actions of n samples of gauge configurations

\sphinxlineitem{Return type}
\sphinxAtStartPar
list of double

\end{description}\end{quote}

\end{fulllineitems}



\renewcommand{\indexname}{Python Module Index}
\begin{sphinxtheindex}
\let\bigletter\sphinxstyleindexlettergroup
\bigletter{a}
\item\relax\sphinxstyleindexentry{analytical\_reject}\sphinxstyleindexpageref{analytical_reject:\detokenize{module-analytical_reject}}
\indexspace
\bigletter{e}
\item\relax\sphinxstyleindexentry{ECMC}\sphinxstyleindexpageref{ECMC:\detokenize{module-ECMC}}
\indexspace
\bigletter{g}
\item\relax\sphinxstyleindexentry{gauge\_su3}\sphinxstyleindexpageref{gauge_su3:\detokenize{module-gauge_su3}}
\indexspace
\bigletter{n}
\item\relax\sphinxstyleindexentry{numerical\_reject}\sphinxstyleindexpageref{numerical_reject:\detokenize{module-numerical_reject}}
\end{sphinxtheindex}

\renewcommand{\indexname}{Index}
\printindex
\end{document}
